%Listing (Eigene Definition)


\usepackage{listings}

\lstset{%
	numbers=left,            % Zelennummern links
	commentstyle=\usefont{T1}{pcr}{m}{sl}\color{DarkGreen}, %test
	breaklines=true,
	frameround=tttt,
	frame=single,
	rulecolor=\color{black},
	numbersep=10pt,
	morekeywords={},																				%test
	keywordstyle=\color{blue},
	stepnumber=1,            % Jede Zeile nummerieren.
	numberstyle=\tiny,       % Zeichengr�sse 'tiny' f�r die Nummern.
	breaklines=true,         % Zeilen umbrechen wenn notwendig.
	breakautoindent=true,    % Nach dem Zeilenumbruch Zeile einr�cken.
	postbreak=\space,        % Bei Leerzeichen umbrechen.
	tabsize=2,               % Tabulatorgr�sse 2
	basicstyle=\ttfamily\footnotesize, % Nichtproportionale Schrift, klein f�r den Quellcode
	showspaces=false,        % Leerzeichen nicht anzeigen.
	showstringspaces=false,  % Leerzeichen auch in Strings ('') nicht anzeigen.
	extendedchars=true,      % Alle Zeichen vom Latin1 Zeichensatz anzeigen.
	stringstyle=\color{mauve},
	backgroundcolor=\color{highlight}, % Hintergrundfarbe des Quellcodes setzen.
}

\usepackage{paralist}
