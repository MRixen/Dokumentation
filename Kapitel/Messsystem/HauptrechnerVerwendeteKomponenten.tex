\subsection{Verwendete Komponenten}
\label{kap:HauptrechnerVerwendeteKomponenten}
Die Bestandteile des Hauptrechners sind ein Raspberry-Pi 2 mit Wlan-Adapter und dem Betriebsystem Windows 10 IoT, sowie einer Energiequelle (Akku von Anker) mit XXX mAh.

\begin{itemize}
	\item \textbf{Raspberry-Pi 2}
	
	Der Raspberry Pi 2 (s. Abbildung \ref{fig:Raspberry}) ist in seiner zweiten Version ein stabiles System, welches in Verbindung mit Windows IoT mit Csharp programmiert werden kann. Die Programme entsprechen den Anwendungen wie man sie von Smartphones kennt.
	\newline
	Die zweite Version beinhaltet (im Vergleich zur Version 3) keinen Wlan-Adapter, weshalb dieser zusätzlich erworben werden muss.
	
	%\begin{figure}[H]
	%	\centering
	%	\includegraphics[width=0.7\linewidth]{Bilder/Raspberry}
	%	\caption[Raspberry Pi 2]{Raspberry Pi 2}
	%	\label{fig:Raspberry}
	%\end{figure}

	\item \textbf{Wlan-Adapter}
	
	Bei der Wahl eines Wlan-Adapters muss die Kompatibilität zu Windows-IoT berücksichtigt werden, da nicht für jedes Gerät entsprechende Treiber zur Verfügung stehen. Aus diesem Grunde liegt die Wahl bei dem preisgünstigen Wlan-Adapter XXXX (s. Abbildung \ref{fig:Adxl345}).
	
	%\begin{figure}[H]
	%	\centering
	%	\includegraphics[width=0.7\linewidth]{Bilder/WlanAdapter}
	%	\caption[Wlan Adapter]{Wlan Adapter}
	%	\label{fig:WlanAdapter}
	%\end{figure}
	
	\item \textbf{Akku von Anker}
	
	Für die Energieversorgung (s. Abbildung \ref{fig:Anker}) wird ein Akku benötigt, welcher eine hohe Kapazität aufweist und genügend Strom liefert, damit das System ordnungsgemäß funktioniert. Bspw. benötigt der Raspberry Pi 2 2A, um einen fehlerfreien Betrieb zu gewährleisten (mit zusätzlich angeschlossener Hardware, wie z.B. der Wlan-Adapter).  
	\newline
	Der Akku XXX von Anker ist für diese Anforderungen wie geschaffen, da dieser XXXX mAh aufweist und 2 USB-Ports mit jeweils 2A zur Verfügung stellt.
		
	%\begin{figure}[H]
	%	\centering
	%	\includegraphics[width=0.7\linewidth]{Bilder/Anker}
	%	\caption[Akku Anker]{Akku Anker}
	%	\label{fig:Anker}
	%\end{figure}

	
\end{itemize}