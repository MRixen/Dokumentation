\subsection{Pre-Release version}
\label{kap:McpExecutorPreReleaseversion}
Bei der Pre-Realease-Version der Mcp-Executor-Einheit wird die Beta-Version zunächst bewertet und alle negativen Aspekte aufgezeigt und möglcihe Lösungen erarbeitet. Die zu verbessernden Punkte sind in Tabelle \ref{tab:McpExecutorEinheitNegAspekte} aufgelistet:

% INSERT TABLE WITH NEGATIVE ASPECTS
\begin{table}[H]
	\caption{Optimierung der Beta-Version}\label{tab:McpExecutorEinheitNegAspekte}
	\centering
	\begin{tabular}{|p{5cm}|p{5cm}|}
		\hline
		\textbf{Negativer Aspekt} & \textbf{Lösung} \\
		\hline
		Spalt zwischen Abdeckblech u. Gehäuse & Höhe um 2mm verringern \\
		\hline
		Gehäuse ist nach außen gebogen & Breite um 1mm verringern \\
		\hline
		Rxd, Txd, Reset Leitungen nicht befestigt & ANschlüsse als Stecker ausführen \\
		\hline
		VCC und GND nicht auf gleicher Seite wie Bus-Anschluss & Anschlüsse seitlich aus Gehäuse herausführen \\
		\hline
		Osc1, Osc2 Leitungen sind eingeklemmt & Leitungen seitlich aus MCU herausführen \\
		\hline
		Kunststoffabdeckung biegt sich durch & AL-Platte verwenden \\
		\hline
		VCC, GND Anschluss nicht fixiert & Platine verlängern und in Gehäuse klemmen \\
		\hline
		Platinen unzureichend in Richtung des Abdeckblechs fixiert & Puffermatieral einbringen \\
		\hline
	\end{tabular} 
\end{table}

Nach dem Ausbessern dieser Punkte zeigt sich die in Abbildung \ref{fig:McpExecutorRelease} dargestellte Einheit. Da auch an dieser Version verschiedene Punkte verbesserungswürdig sind handelt es sich hierbei um eine Pre-Release Version.

%\begin{figure}[H]
%	\centering
%	\includegraphics[width=0.7\linewidth]{Bilder/McpExecutorPreRelease}
%	\caption[Pre-Release-Version einer Mcp-Executor-Einheit]{Pre-Release-Version einer Mcp-Executor-Einheit}
%	\label{fig:McpExecutorPreRelease}
%\end{figure}