\subsection{Beschreibung HMI - Main}
\label{kap:ClientGraphProgramm}
Die HMI gliedert sich in mehrere Bereiche, in denen verschiedene Funktionalitäten bereitgestellt werden. Diese finden im folgenden Abschnitt Erläuterung. 

\begin{figure}[H]
\centering
\includegraphics[width=1.0\linewidth]{Bilder/ClientGraph_Gui}
\caption[Client Graph HMI]{Client Graph HMI}
\label{fig:clientgraphgui}
\end{figure}

\begin{enumerate}
	\item \textbf{Aufnahme starten} \\
	Der grüne Balken dient als Indikator, ob Messwerte empfangen werden (grün: Messung kann gestartet werden, rot: Es werden keine Daten empfangen).\\
	Durch das Betätigen des Buttons \textit{Record to db} wird eine Messung gestartet bis diese abgebrochen oder die max. Anzahl an Samples erreicht wird.\\
	Das Label \textit{Saved rows} zeigt die aktuelle Anzahl aufgenommener Messwerte an.\\
	Im Textfeld \textit{Context} kann eine Bezeichnung der Messwerte eingegeben werden (die exportierten Textdateien tragen diesen Namen).
	
	\item \textbf{Import / Export und Kalibrierung}\\
	Mit dem Betätigen des Buttons \textit{Export...} erfolgt das Exportieren der erfassten Sensordaten in eine CSV-Datei. Jeder Sensor generiert eine eigene Datei mit dem unter \textit{COntext} angegebenen Namen.\\
	Durch das Betätigen des Buttons \textit{Import...} kann eine Textdatei (CSV-Format) eingelesen und in die Datenbank geschrieben werden.\\
	Durch das Betätigen des Buttons \textit{Calibrate Sensors} erfolgt das Kalibrieren aller Sensoren (s. Beschreibung in \ref{kap:Erfassungsprogramm}). Durch das Setzen des Häkchens \textit{Show ...} erfolgt das Darstellen der kalibrierten Sensorwerte.
	
	\item \textbf{Aufnahmezeit}\\
	In das Textfeld \textit{Max. samples} wird die aufzunehmende Anzahl an Messwerten eingetragen. Danach stoppt die Messung automatisch. Standardmäßig sind 3500 Samples eingetragen, da bei einer zu großen Anzahl eine \textit{Out of memory Exception} geworfen wird.
	
	\item \textbf{Settings}\\
	Durch das Setzen des Häkchens bei \textit{Show graph} wird ein grafische Visualisierung der Sensorwerte dargestellt.\\
	Durch das Setzen des Häkchens bei \textit{Show database} werden die Einträge der Datenbank dargestellt.
	
	\item \textbf{Information}\\
	In dem Textfeld erfolgt das Darstellen wichtiger Systemereignisse (z.B. Fehler), die durch das Betätigen des Buttons \textit{Clear} gelöscht werden können.
\end{enumerate}

